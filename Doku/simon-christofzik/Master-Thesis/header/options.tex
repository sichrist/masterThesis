% !TEX root = ../thesis.tex

\usepackage{todonotes}
\usepackage{enumitem}
\usepackage{multirow}
\usepackage{array}
\usepackage{graphicx}
\usepackage{xcolor}
\usepackage{tikz}

\newcommand\MyBox[3]{
	\fcolorbox{white}{#3}{
		\lower1.0em
		\vbox to 2.7cm{\vfil
			\hbox to 2.7cm{\hfil\parbox{2.3cm}{\bfseries\centering#1\\\centering#2}\hfil}
			\vfil}%
	}
}%

\definecolor{applegreen}{rgb}{0.13, 0.55, 0.13}

\newcommand\MyThinnerBox[2]{
	\fcolorbox{white}{#2}{
		\lower1.0em
		\vbox to 2.7cm{\vfil
			\hbox to 1.3cm{\hfil\parbox{1.0cm}{\bfseries\centering#1}\hfil}
			\vfil}%
	}
}%

\newcommand\MySmallerThinnerBox[2]{
	\fcolorbox{white}{#2}{
		\lower0.5em
		\vbox to 1.3cm{\vfil
			\hbox to 1.3cm{\hfil\parbox{1.0cm}{\bfseries\centering#1}\hfil}
			\vfil}%
	}
}%

\newcommand\MySmallerBox[2]{
	\fcolorbox{white}{#2}{
		\lower0.5em
		\vbox to 1.3cm{\vfil
			\hbox to 2.7cm{\hfil\parbox{2.3cm}{\bfseries\centering#1}\hfil}
			\vfil}%
	}
}%



\newcolumntype{L}[1]{>{\raggedright\arraybackslash}p{#1}} % linksbündig mit Breitenangabe
\newcolumntype{C}[1]{>{\centering\arraybackslash}p{#1}} % zentriert mit Breitenangabe
\newcolumntype{R}[1]{>{\raggedleft\arraybackslash}p{#1}} % rechtsbündig mit Breitenangabe

\newenvironment{condright}
{\par\vspace{\abovedisplayskip}\noindent\begin{tabular}{>{$}l<{$} @{${}={}$} l}}
	{\end{tabular}\par\vspace{\belowdisplayskip}}

\newenvironment{condleft}
{\par\vspace{\abovedisplayskip}\noindent\begin{tabular}{l @{${}={}$} >{$}l<{$}}}
	{\end{tabular}\par\vspace{\belowdisplayskip}}

\newcommand*\rot{\rotatebox{90}}

\newcommand*{\footurl}[1]{\footnote{\url{#1}}}


\def\PointNetpp{{PointNet\nolinebreak[4]\hspace{-.05em}\raisebox{.4ex}{\tiny\bf ++}}}

\usepackage{caption}
\captionsetup{format=hang}
\usepackage{float}
\restylefloat{table}


 \usepackage[notquote]{hanging}
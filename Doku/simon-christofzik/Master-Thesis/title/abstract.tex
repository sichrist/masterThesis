% !TEX root = ../thesis.tex
\thispagestyle{plain}
\cleardoublepage
\phantomsection
\addcontentsline{toc}{chapter}{Abstract}
\vspace*{11pt}
\begin{center}
	{\LARGE \textbf{\textsf{Abstract}}}
\end{center}

\bigskip
\begin{center}
	\begin{tabular}{p{3.2cm}p{9.6cm}}
		Titel: & \thema \\
		 & \\
		Masterkandidat: & \autor \\
		 & \\
		Prüfer: & \firmaA \\ & \firmaB \\[1.1ex] & \prueferA  \\[.5ex]
		 &  \prueferB \\
		 & \\
		Abgabedatum: & \abgabedatum \\
		 & \\
		Schlagworte: & \schlagworte \\
		 & \\
	\end{tabular}
\end{center}

\bigskip

\noindent

In dieser Arbeit wird mithilfe von Optimierungsverfahren Parameter für Schwarmmodelle geschätzt.
In der gängigen Literatur zu Schwarmverhalten stößt man immer wieder auf agentenbasierte Modelle.
Eines der bekanntesten Modelle im Bezug auf agentenbasierte Modellierung ist das Boidsmodell, welches von Craig Reynolds vor über 30 Jahren publiziert wurde. Ziel dieser Arbeit ist es, Zustände, welche in echten Schwärmen zu beobachten sind, durch das Boidsmodell zu imitieren. Dazu werden die Parameter des Schwarmmodelles an Datensätze von Schwärmen angepasst. Des Weiteren wird ein neues Modell vorgestellt, welches im Laufe der Arbeit entstanden ist. Dieses Modell wird ebenfalls per Approximation der Parameter an die Datensätze angepasst. Der direkte Vergleich der erreichten Zustände der Modelle und der Zustände des realen Datensatzes gibt hierbei Aufschluss über die Modellannahmen.
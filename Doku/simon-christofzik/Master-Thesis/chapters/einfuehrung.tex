% !TEX root = ../thesis.tex

\chapter{Einleitung} % (fold)
\label{cha:einleitung}

Schwarmverhalten ist eine der grundlegenden Verhaltensmuster in unserer Welt. 
Aufgaben, die für das Individuum unmöglich sind, werden durch die Zusammenarbeit vieler Individuen erst möglich.
Faszinierend ist, dass der einzelne Teilnehmer des Schwarms das große Ganze nicht sieht, durch die Kooperation dennoch etwas Großes zustande kommt. Einfache Lebensformen wie beispielsweise Ameisen verbringen im Kollektiv bemerkenswerte Aufgaben.
Das Nest verschiedener Ameisenspezies sind kunstvoll in ihrer Architektur und dennoch funktional.
Die Blattschneiderameise \textit{Atta vollenweideri} errichtet Bauten, deren Belüftung den Gehalt von Kohlenstoffdioxid im Bau reguliert. Der Ameisenbau beinhaltet Kammern, in denen Nahrungsmittel wie Pilze heranwachsen \cite{HalbothRoces2017}. 
Es ist bemerkenswert, dass diese Kreaturen auf der Basis von lokalen Informationen und selbstständiger Arbeitsweise Aufgaben vollbringen, ohne eine zentrale Steuereinheit zu besitzen. Auch in uns Menschen stecken Systeme, die nach dem Muster des Schwarmverhaltens schützen und somit am Leben halten, wie zum Beispiel unser Immunsystem. Neutrophils spielen eine große Rolle in der frühen Bekämpfung von Krankheitserregern. Durch das Aussenden von Chemotaxis wird die Immunabwehr angeregt und koordiniert dadurch das Schwarmverhalten von Nachbarzellen \cite{doi:10.1126/science.abe7729}.

Die Bionik beschäftigt sich mit dem Übertragen von Phänomenen aus der Natur in die Technik. Eines der bekanntesten Beispiels ist das der Gebrüder Wright, die durch das beobachten von Vögeln ihre Flugzeugprototypen optimierten \cite{wrightbrothers}.
Auch die zuvor erwähnte Ameisenkollonie diente bei der Entwicklung des Ameisenalgorithmus als Vorbild für die Suche nach einem kürzesten Weg in einem Graphen.

In den vergangenen Jahren ist die Entwicklung des autonomen Fahrens immer stärker vorangeschritten. Das autonome Fahren verspricht viele Vorteile, wie das selbstständige Transportieren von Personen, wodurch diese nicht mit Fahren beschäftigt sind und sich produktiveren Angelegenheiten widmen können. Auch effizientere Fahrtwege und das vermeiden von Staus können daraus hervorgehen.
Hierdurch kann ein geringerer Kraftstoffverbrauch resultieren, was unserem Klima zu gute kommen kann.
Die Bezeichnung autonomes Fahren impliziert, dass Autos keiner zentrale Steuereinheit folgen, sondern selbstständig Entscheidungen treffen. Diese müssen im Einklang mit den lokalen Einflüssen wie Wetterbedingungen, Fahrbahnbeschaffenheit und insbesondere der anderen Verkehrsteilnehmer getroffen werden. Doch wie kann das autonome Fahren effizient und sicher gestaltet werden?
Die Antwort scheint wie so oft in der Tierwelt zu liegen und dem Studieren von Schwarmverhalten.

Das Verständnis von Phänomenen der Natur wird üblicherweise durch Modelle geschaffen, die das Phänomen erklären sollen.
Schwarmverhalten stellt hier keine Ausnahme dar. Durch das Formulieren von Schwarmmodellen können Eigenschaften eines Schwarms genauer betrachtet werden und auch Vorhersagen über das Verhalten eines Schwarms gemacht werden.

In dieser Arbeit wird sich mit einem gängigen Schwarmmodell auseinandergesetzt. Das Ziel ist, mit Hilfe eines Modells Verhaltensmuster eines echten Schwarms abzubilden, indem Parameter für die vorgestellten Modelle geschätzt werden. Dazu wird das Modell auf Realdaten justiert und das Verhalten des Modells dem des echten Schwarms gegenübergestellt. Aus den Ideen und Beobachtungen des gängigen Modells wird zudem ein eigenes Modell abgeleitet.
Dadurch können zwei Modelle gegenübergestellt und verglichen werden.

Der Aufbau der Arbeit beginnt mit den Grundlagen, die notwendig sind, um dieses Unterfangen zu bewältigen.
Die grundlegenden Eigenschaften eines Schwarms werden zu Beginn angeführt. Darauf folgen Optimierungsverfahren, welche es ermöglichen, Daten und Modell in Einklang zu bringen. Anschließend folgt ein Exkurs über die Betrachtung von Bewegungen, wodurch Videoaufnahmen von Schwärmen in eine Form gebracht werden, um diese verarbeiten zu können. Die mathematische Definition der Modelle wird in dem darauffolgenden Kapitel eingeführt. Zudem folgt die Definition der Zustände, die dem Datensatz innewohnen.
Das vorletzte Kapitel widmet sich den Experimenten, in denen die Pfade der Realdaten extrahiert werden und die Modelle an diese angepasst werden. Schlussendlich werden die Ergebnisse im Ausblick für weitere Arbeiten dargestellt.


% chapter einleitung (end)
